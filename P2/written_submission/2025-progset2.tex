\documentclass[11pt]{article}
% \pagestyle{empty}

\setlength{\oddsidemargin}{-0.25 in}
\setlength{\evensidemargin}{-0.25 in}
\setlength{\topmargin}{-0.9 in}
\setlength{\textwidth}{7.0 in}
\setlength{\textheight}{9.0 in}
\setlength{\headsep}{0.75 in}
\setlength{\parindent}{0.3 in}
\setlength{\parskip}{0.1 in}
\usepackage{epsf}

\def\O{\mathop{\smash{O}}\nolimits}
\def\o{\mathop{\smash{o}}\nolimits}
\newcommand{\e}{{\rm e}}
\newcommand{\R}{{\bf R}}
\newcommand{\Z}{{\bf Z}}

\begin{document}
	
	\section*{CS 1240 Programming Assignment 2: Spring 2025}
 		
	\textbf{Your name(s) (up to two):} 
		
	\textbf{Collaborators:} (You shouldn't have any collaborators but the up-to-two of you, but tell us if you did.)

	\textbf{No. of late days used on previous psets: }\\ \indent
	\textbf{No. of late days used after including this pset: }

Homework is due Wednesday 2025-04-02 at 11:59pm ET. You are allowed up to {\bf twelve} late days through the semester, but the number of late days you take on each assignment must be a nonnegative integer at most {\bf two}. 

\noindent {\bf {\Large Overview}:} 

(Much of this progset will make a lot more sense after lecture on Monday, March 3. But here's the progset in advance anyway. If you are familiar with matrix multiplication you can start coding up standard matrix multiplication now. If you are not, now is a great time to brush up before lecture on Monday!!)

Strassen's divide and conquer matrix multiplication algorithm for $n$
by $n$ matrices is asymptotically faster than the conventional
$\O(n^3)$ algorithm.  This means that for sufficiently large values of
$n$, Strassen's algorithm will run faster than the conventional
algorithm.  For small values of $n$, however, the conventional
algorithm may be faster.  Indeed, the textbook Algorithms in C (1990
edition) suggests that $n$ would have to be in the thousands before
offering an improvement to standard multiplication, and ``Thus the
algorithm is a theoretical, not practical, contribution.''  Here we
test this armchair analysis.

Here is a key point, though (for any recursive algorithm!).  Since
Strassen's algorithm is a recursive algorithm, at some point in the
recursion, once the matrices are small enough, we may want to switch
from recursively calling Strassen's algorithm and just do a
conventional matrix multiplication.  That is, the proper way to do
Strassen's algorithm is to not recurse all the way down to a ``base
case'' of a 1 by 1 matrix, but to switch earlier and use conventional
matrix multiplication.  That is, there's no reason to do a ``base
case'' of a 1 by 1 matrix; it might be faster to use a larger-sized
base case, as conventional matrix multiplication might be faster up to
some reasonable size.  Let us define the {\em cross-over point}
between the two algorithms to be the value of $n$ for which we want to
stop using Strassen's algorithm and switch to conventional matrix
multiplication.  The goal of this assignment is to implement the
conventional algorithm and Strassen's algorithm and to determine their
cross-over point, both analytically and experimentally.  One important
factor our simple analysis will not take into account is memory
management, which may significantly affect the speed of your
implementation.

\noindent {\bf {\Large Tasks:}}

\begin{enumerate}

\item Assume that the cost of any single arithmetic operation (adding,
subtracting, multiplying, or dividing two real numbers) is 1, and that
all other operations are free.  Consider the following variant of
Strassen's algorithm: to multiply two $n$ by $n$ matrices, start using
Strassen's algorithm, but stop the recursion at some size $n_0$, and
use the conventional algorithm below that point.  You have to find a
suitable value for $n_0$ -- the cross-over point.  Analytically
determine the value of $n_0$ that optimizes the running time of this
algorithm in this model.  (That is, solve the appropriate equations, somehow,
numerically.)
This gives a crude estimate for the cross-over point between
Strassen's algorithm and the standard matrix multiplication algorithm.

\medskip Note: there are multiple ways to handle matrices of odd dimension, which may give different values of $n_0$. Any of them will be valid as long as the approach is properly specified, but the one that we expect most students to come up with involves padding with 0's (refer to the hint at the bottom of the document).
\item Implement your variant of Strassen's algorithm and the standard
matrix multiplication algorithm to find the cross-over point
experimentally.  Experimentally optimize for $n_0$ and compare the
experimental results with your estimate from above.  Make both
implementations as efficient as possible.  The actual cross-over
point, which you would like to make as small as possible, will depend
on how efficiently you implement Strassen's algorithm.  Your implementation
should work for any size matrices, not just those whose dimensions are a power
of 2.

To test your algorithm, you might try matrices where each entry is
randomly selected to be 0 or 1; similarly, you might try matrices
where each entry is randomly selected to be 0, 1 or 2, or instead 0,
1, or $-1$.  We will test on integer matrices, possibly of this form.
(You may assume integer inputs.)  You need not try all of these, but
do test your algorithm adequately.

\item Triangle in random graphs:  Recall that you can represent the
adjacency matrix of a graph by a matrix $A$.  Consider an undirected
graph.  It turns out that $A^3$ can be used to determine the number
of triangles in a graph: the $(ij)$th entry
in the matrix $A^2$ counts the paths from $i$ to $j$ of lenth two,
and the $(ij)$th entry in the matrix $A^3$ counts the path from $i$ to
$j$ of length 3.  To count the number of triangles in in graph, we can
simply add the entries in the diagonal, and divide by 6.  This is because
the $j$th diagonal entry counts the number of paths of length 3 from $j$
to $j$.  Each such path is a triangle, and each triangle is counted 6 times
(for each of the vertices in the triangle, it is counted once in each direction).  

Create a random graph on $1024$ vertices where each edge is included with probability $p$ for each of the following values of $p$:  $p = 0.01, 0.02, 0.03, 0.04,$ and $0.05$.  Use your (Strassen's) matrix multiplication code to count the number of triangles in each of these graphs, and compare it to the expected number of triangles, which is ${1024 \choose 3} p^3$.  Create a chart showing your results compared to the expectation.  

\end{enumerate}

\noindent {\bf {\Large Code setup:}} 

60\% of the score for problem set 2 is determined by an autograder. You can submit code to the autograder on Gradescope repeatedly; only your latest submission will determine your final grade. We support the following programming languages: Python3, C++, C, Java, Go; if you want to use another language, please contact us about it. 

 {\bf {\Large Option 1: Single-source file:}} 

In this option, you can submit a single source file. Please make sure to NOT submit a makefile/Makefile if you elect to use this option, as it confuses the autograder. Please ensure that you have exactly one of the following files in your directory if you choose to use this option: 
\begin{enumerate}
\item strassen.py - for python. In this case, we will run

python3 strassen.py $<$args$>$
\item strassen.c - for C. In this case, we will run 

gcc -std=c11 -O2 -Wall -Wextra strassen.c -o strassen -lm -lpthread

./strassen $<$args$>$
\item strassen.cpp - for C++. In this case, we will run 

g++ -std=c++17 -O2 -Wall -Wextra strassen.cpp -o strassen -lm -lpthread

./strassen $<$args$>$

\item strassen.java / Strassen.java - for Java. In this case, we will run 

javac strassen.java (javac Strassen.java)

java -ea strassen $<$args$>$ (java -ea Strassen $<$args$>$) 
\item strassen.go - for Go. In this case, we will run 

go build strassen.go

go run strassen.go $<$args$>$
\end{enumerate}
 {\bf {\Large Option 2: Makefile:}} 

In this option, you submit a makefile (either Makefile or makefile). In this case, the autograder first runs make. Then, it identifies the language and runs the corresponding command above.

Your code should take three arguments: a flag, a dimension, and an input file:

\noindent \$ ./strassen 0 dimension inputfile 

The flag 0 is meant to provide you some flexibility; you may use other
values for your own testing, debugging, or extensions.  The dimension,
which we refer to henceforth as $d$, is the dimension of the matrix
you are multiplying, so that 32 means you are multiplying two 32 by 32
matrices together.  The inputfile is an ASCII file with $2d^2$ integer
numbers, one per line, representing two matrices $A$ and $B$; you are
to find the product $AB = C$.  The first integer number is matrix entry
$a_{0,0}$, followed by $a_{0,1},a_{0,2},\ldots,a_{0,d-1}$; next comes
$a_{1,0},a_{1,1}$, and so on, for the first $d^2$ numbers.  The next
$d^2$ numbers are similar for matrix $B$.

Your program should put on standard output (in C: printf, cout, System.out,
etc.) a list of the values of the {\em diagonal entries}
$c_{0,0},c_{1,1},\ldots,c_{d-1,d-1}$, one per line, including a
trailing newline.  The output will be checked by a script -- add no
clutter.  (You should not output the whole matrix, although of course
all entries should be computed.)

The inputs we present will be small integers, but you should make sure
your matrix multiplication can deal with results that are up to 32 bits.

Do not turn in an executable.

\noindent {\bf {\Large What to hand in:}} 

As before, you may work in pairs, or by yourself.  Hand in a project
report (on paper) describing your analytical and experimental work
(for example, carefully describe optimizations you made in your
implementations).  Be sure to discuss the results you obtain, and try
to give explanations for what you observe.  How low was your
cross-over point?  What difficulties arose?  What types of matrices did you
multiply, and does this choice matter?

Your grade will be based primarily on the correctness of your program,
the crossover point you find, your interpretation of the
data, and your discussion of the experiment.

\noindent {\bf {\Large Hints:}} 

It is hard to make the conventional algorithm inefficient; however,
you may get better caching performance by looping through the
variables in the right order (really, try it!).  For Strassen's algorithm:
\begin{itemize}
\item Avoid excessive memory allocation and deallocation.  This requires
some thinking.
\item Avoid copying large blocks of data unnecessarily.   This requires
some thinking.
\item Your implementation of Strassen's algorithm should
work even when $n$ is odd!  This requires some additional work, and thinking.
(One option is to pad with 0's;  how can this be done most effectively?)
However, you may want to first get it to work when $n$ is a power of 2 --
this will get you most of the credit -- and then refine it to work for
more general values of $n$.  
\end{itemize}
\end{document}











